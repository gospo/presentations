% TEMPLATE for Usenix papers, specifically to meet requirements of
%  USENIX '05
% originally a template for producing IEEE-format articles using LaTeX.
%   written by Matthew Ward, CS Department, Worcester Polytechnic Institute.
% adapted by David Beazley for his excellent SWIG paper in Proceedings,
%   Tcl 96
% turned into a smartass generic template by De Clarke, with thanks to
%   both the above pioneers
% use at your own risk.  Complaints to /dev/null.
% make it two column with no page numbering, default is 10 point

% Munged by Fred Douglis <douglis@research.att.com> 10/97 to separate
% the .sty file from the LaTeX source template, so that people can
% more easily include the .sty file into an existing document.  Also
% changed to more closely follow the style guidelines as represented
% by the Word sample file. 

% Note that since 2010, USENIX does not require endnotes. If you want
% foot of page notes, don't include the endnotes package in the 
% usepackage command, below.

% This version uses the latex2e styles, not the very ancient 2.09 stuff.

% Updated July 2018: Text block size changed from 6.5" to 7"

\documentclass[letterpaper,twocolumn,10pt]{article}
\usepackage{int,graphicx,endnotes}
\begin{document}

%don't want date printed
\date{}

%make title bold and 14 pt font (Latex default is non-bold, 16 pt)
\title{\Large \bf Implementing the Next Generation of Network
Telemetry Technologies}

%for single author (just remove % characters)
\author{
{\rm Andy Gospodarek}\\
Broadcom Corporation
% copy the following lines to add more authors
% \and
% {\rm Name}\\
%Name Institution
} % end author

\maketitle

\subsection*{Abstract}
Current network monitoring and telemetry applications require host-based
collectors across all nodes in a network (both on servers/hypervisors
and traditional switches and routers). These can be effective solutions,
but just as datacenter deployment patterns have evolved new technology
to track traffic as it moves through the network had emerged. Newer
specifications like INT~\cite{INT} and IFA~\cite{IFA} propose standards
to add metadata to packets or clone and add metadata as they flow
through a network to allow collectors/agents to gather data at the
network edges. Hardware that supports INT/IFA can add metadata
automatically with application/flowlevel/virtual-port granularity which
allows more detailed network monitoring and assurance to customers that
service levels for applications are being met.

\section{Introduction to INT and IFA}

Introduction to INT and IFA network elements including currently proposed on-wire frame formats

\subsection{INT Frame Formats}

\subsection{IFA Frame Formats}

\section{Hardware Requirements}

Hardware requirements/implementations for handling IFA/INT today and the challenges facing those
wanting to implement them in hardware and software.

\section{Possible Configuration Methods}

Proposals for configuring INT/IFA in both software dataplane and
hardware dataplane environments on supported hardware.

\subsection{Host-based configuration}

\subsection{Network-based configuration}

\section{Risks and Rewards}

Assessing and minimizing risk associated with the maintenance of
deploying an early draft of a standard.

\section{Acknowledgments}

Special thanks to all those invoved in IFA and INT specs as well as
those at Broadcom who did not particiapte in the writing of the
standard, but did provide code, feedback, etc during this process.

{\normalsize \bibliographystyle{acm}
\bibliography{int}}


\theendnotes

\end{document}

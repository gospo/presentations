% TEMPLATE for Usenix papers, specifically to meet requirements of
%  USENIX '05
% originally a template for producing IEEE-format articles using LaTeX.
%   written by Matthew Ward, CS Department, Worcester Polytechnic Institute.
% adapted by David Beazley for his excellent SWIG paper in Proceedings,
%   Tcl 96
% turned into a smartass generic template by De Clarke, with thanks to
%   both the above pioneers
% use at your own risk.  Complaints to /dev/null.
% make it two column with no page numbering, default is 10 point

% Munged by Fred Douglis <douglis@research.att.com> 10/97 to separate
% the .sty file from the LaTeX source template, so that people can
% more easily include the .sty file into an existing document.  Also
% changed to more closely follow the style guidelines as represented
% by the Word sample file. 

% Note that since 2010, USENIX does not require endnotes. If you want
% foot of page notes, don't include the endnotes package in the 
% usepackage command, below.

% This version uses the latex2e styles, not the very ancient 2.09 stuff.

% Updated July 2018: Text block size changed from 6.5" to 7"

\documentclass[letterpaper,twocolumn,10pt]{article}
\usepackage{int,graphicx,endnotes}
\setcounter{secnumdepth}{4}
\begin{document}

%don't want date printed
\date{}

%make title bold and 14 pt font (Latex default is non-bold, 16 pt)
\title{\Large \bf Moonshot: Implementing the Next Generation of Network
Telemetry Technologies}

%for single author (just remove % characters)
\author{
{\rm Andy Gospodarek}\\
Broadcom Corporation
% copy the following lines to add more authors
% \and
% {\rm Name}\\
%Name Institution
} % end author

\maketitle

\subsection*{Abstract}
Current network monitoring and telemetry applications require host-based
collectors across all nodes in a network (both on servers/hypervisors
and traditional switches and routers). These can be effective solutions,
but just as datacenter deployment patterns have evolved new technology
to track traffic as it moves through the network had emerged. Newer
specifications like Inband Network Telemetry (INT)~\cite{INT} and Inband
Flow Analyzer (IFA)~\cite{IFA} propose standards to add metadata to
packets or clone and add metadata as they flow through a network to
allow collectors/agents to gather data at the network edges. Hardware
that supports INT/IFA can add metadata automatically with
application/flowlevel/virtual-port granularity which allows more
detailed network monitoring and assurance to customers that service
levels for applications are being met.

\section{Introduction to Network Telemetry}

INT and IFA are all designed create a generic method of reporting and
collecting network state information on individual flows as the packets
traverse a network.  This allows for collection of data from individual
hosts or applications as frames that are part of those flows are marked
with telemetry headers as they entry a telemetry domain.  Network
devices can interpret telemetry header fields as \textit{telemetry
instructions} and a capable device will update packet headers and
header-fields \textit{In-Situ} -- as a frame traverses the network.
Marking frames as they travel through a network allows detailed
reporting of the exact data-plane used by packets on the network as well
as enables real-time feedback loops and event detection.  This
information can also be sent to an external collector for
post-processing if desired.  

\subsection{Network Telemetry Components}

Despite using slightly different nomenclature, the fundamental
components of the two main telemetry technolgies covered in the paper
are similar.

\subsubsection{Source or Initiator Node}
This is a trusted entity that creates the initial telemetry header and
places it into packets that are transmitted.

\subsubsection{Transit Hop or Transit Node}
Any network element that adds telemetry metadata to a packet that
that contains supported telemetry instructions.

\subsubsection{Sink or Terminating Node}
This is a trusted entity that removes telemetry headers from frames to
make the existence of the the headers transparent to applications.  This
trusted entity will use the headers and other local configuration to
determine if information needs to be sent to a collector.

\subsubsection{Collector}
An application that will receive telemetry data collected by a Sink or
Terminating node.

\subsubsection{Typical Packet Flow}

Below is a diagram of a typical packet path of a network flow and IFA
flow through the components described in the previous section.  In this
case the IFA flow is a sample of the flow, so two frames travel between
the Initiator Node and the Terminating Node.

This was adapted from the latest IFA specification and the time fo this
writing:
\tiny
\begin{center}
\begin{verbatim}
                              +----------+
                              |          |
                              |Collector |<-------------+
                              |          |              |
                              +----------+              |
                                                        |
                                                        |
                                                        |
                                                        |
                                                        |
                                                        |
                                                        | 
                                                        |
     +--------------+        +------------+        +----+-----------+
     |Initiator Node|        |Transit Node|        |Terminating Node|
flow |   +------+   |  flow  |  +------+  |  flow  |     +------+   | flow
---->|   | IFA  |   |------->|  | IFA  |  |------->|     | IFA  |   |---->
     |   +------+   |IFA flow|  +------+  |IFA flow|     +------+   |
     +--------------+        +------------+        +----------------+
\end{verbatim}
\end{center}
\normalsize

In the case where an Initiator and Terminating nodes are switches or
other forwarding elements on a network, flows could originate from an
external device and exit to another device.  Initiator and Terminating
Nodes could also be servers with supported hardware and software stacks.
In that case flows may originate or teminate on the IFA/INT node rather
than originating/terminating from an external device.

This paper does not intend to cover the full scope of each telemetry
technology and feature; anyone who would like to learn more should
consider reading the latest INT and IFA specifications.  It will cover
some basic frame formats of each proposal in order to provide context
for an implementation discussion.

\section{Inband Network Telemetry (INT)}

Inband Network Telemetry is a framework suggested by
those interested in using P4 to create a programmable pipeline for
networking forwarding elements.  INT has multiple methods for collecting
information about the network: frames are updated as they traverse the
network or special \textit{probe packets} are used to collect
information about the network.  In addition to defining the frame format
and fields, the latest INT specification also conveniently provides a P4
program specification for INT Transmit.

\subsubsection{INT Frame Format}

The current INT specification describes the following formats as being able to
support additional encapsulation headers to support INT:

\begin{itemize}
\item INT over VXLAN (as VXLAN payload, per GPE extension)
\item INT over Geneve (as Geneve option)
\item INT over GRE (as a shim between GRE header and encapsulated payload)
\item INT over NSH (as NSH payload)
\end{itemize}

Additionally the INT specification also describes how DSCP bits or
\textit{probe markers} can be placed in the payload of packet (after the Layer4
header) to support these packet formats.

\begin{itemize}
\item INT over TCP (as payload)
\item INT over UDP (as payload)
\end{itemize}

Though many datacenter networks use encapsulated traffic (VXLAN, Geneve,
or GRE), the fact that INT does not have native support unencapsulated
traffic (standard IPv4/IPv6 and TCP/UDP) could be an issue for some
deployments.

The specified frame format for an INT IPv4/TCP frame would be as follows:

\begin{center}
\tiny
\begin{verbatim}
       0                   1                   2                   3
       0 1 2 3 4 5 6 7 8 9 0 1 2 3 4 5 6 7 8 9 0 1 2 3 4 5 6 7 8 9 0 1
      +-+-+-+-+-+-+-+-+-+-+-+-+-+-+-+-+-+-+-+-+-+-+-+-+-+-+-+-+-+-+-+-+
      |                        IP Header                              |
      +-+-+-+-+-+-+-+-+-+-+-+-+-+-+-+-+-+-+-+-+-+-+-+-+-+-+-+-+-+-+-+-+
      |                         Layer 4                               |
      +-+-+-+-+-+-+-+-+-+-+-+-+-+-+-+-+-+-+-+-+-+-+-+-+-+-+-+-+-+-+-+-+
      |                      INT Shim Header                          |
      +-+-+-+-+-+-+-+-+-+-+-+-+-+-+-+-+-+-+-+-+-+-+-+-+-+-+-+-+-+-+-+-+
      |                    INT Metadata Header                        |
      +-+-+-+-+-+-+-+-+-+-+-+-+-+-+-+-+-+-+-+-+-+-+-+-+-+-+-+-+-+-+-+-+
      |                         Payload                               |
      +-+-+-+-+-+-+-+-+-+-+-+-+-+-+-+-+-+-+-+-+-+-+-+-+-+-+-+-+-+-+-+-+
\end{verbatim}
\normalsize
\end{center}

Remember that the INT Headers and Payload together are viewed as the full
payload to any non-INT-aware device, so anytime INT headers are added to
a packet any fields that account for the size of the packet or
payload will need to be adjusted.

If the decision to use a reserved DSCP mark (0x17 in this case) to
indicate a packet contained INT headers would cause the IPv4 header to
look like this:

\tiny
\begin{center}
\begin{verbatim}
       0                   1                   2                   3
       0 1 2 3 4 5 6 7 8 9 0 1 2 3 4 5 6 7 8 9 0 1 2 3 4 5 6 7 8 9 0 1
      IPv4 Header:
      +-+-+-+-+-+-+-+-+-+-+-+-+-+-+-+-+-+-+-+-+-+-+-+-+-+-+-+-+-+-+-+-+
      | Ver=4 | IHL=5 | DSCP=0x17 |ECN|          Total Length         |
      +-+-+-+-+-+-+-+-+-+-+-+-+-+-+-+-+-+-+-+-+-+-+-+-+-+-+-+-+-+-+-+-+
      |         Identification        |Flags|      Fragment Offset    |
      +-+-+-+-+-+-+-+-+-+-+-+-+-+-+-+-+-+-+-+-+-+-+-+-+-+-+-+-+-+-+-+-+
      |  Time to Live | Protocol = TCP|         Header Checksum       |
      +-+-+-+-+-+-+-+-+-+-+-+-+-+-+-+-+-+-+-+-+-+-+-+-+-+-+-+-+-+-+-+-+
      |                     Source IPv4 Address                       |
      +-+-+-+-+-+-+-+-+-+-+-+-+-+-+-+-+-+-+-+-+-+-+-+-+-+-+-+-+-+-+-+-+
      |                  Destination IPv4 Address                     |
      +-+-+-+-+-+-+-+-+-+-+-+-+-+-+-+-+-+-+-+-+-+-+-+-+-+-+-+-+-+-+-+-+
\end{verbatim}
\end{center}
\normalsize

The INT specification also outlines suggestions for how to deal with
frames as they grow beyond the MTU, how to deal with false detection of
\textit{probe markers} contained in payload data of non-INT frames, as well as
other deployment scenarios.

\section{Inband Flow Analyzer (IFA)}

The initial IFA specification was drafted later than other initial
telemetry technologies and while similar, it aims to address some of the
shortcomings of INT and IOAM.  One of the main differences is the
ability to send telemetry metadata via a cloned frame rather than via
the original datagram.  

Allowing cloned frames provides benefits over In-Situ modification of
frames.  One benefit of cloning is administrators do not need to be
concerned about frames growing beyond the MTU since there is also
support to allow truncation of frames that are beyond the size of the
MTU.  The IFA specification also indicates that adding metadata to live
traffic is a requirement but this cloning feature is a nice addition to
avoid disruption of PMTU discovery.

The proposed frame/header format was modified significantly from the INT
specification.  The goal was to make the frame format more acceptable to
devices that were not IFA-aware.

\subsubsection{IFA Frame Format}
The IFA spec outlines a significantly different scheme for the location
to telemetry metadata.  From the start IFA aims to interoperate with
unencapsulated IPv4 and IPv6 traffic.  This is accomplished by using the
IPv4 \textit{Protocol} and IPv6 \textit{Next Header} fields to specify
that this frame is an IFA frame.  (There is no current reservation for
IFA protocol, so testing currently uses one of the experimental protocol
numbers.)

The specified frame format for an IFA IPv4/TCP frame would be as follows:
\tiny
\begin{center}
\begin{verbatim}
       0                   1                   2                   3
       0 1 2 3 4 5 6 7 8 9 0 1 2 3 4 5 6 7 8 9 0 1 2 3 4 5 6 7 8 9 0 1
      +-+-+-+-+-+-+-+-+-+-+-+-+-+-+-+-+-+-+-+-+-+-+-+-+-+-+-+-+-+-+-+-+
      |                        IP Header                              |
      +-+-+-+-+-+-+-+-+-+-+-+-+-+-+-+-+-+-+-+-+-+-+-+-+-+-+-+-+-+-+-+-+
      |                       IFA Header                              |
      +-+-+-+-+-+-+-+-+-+-+-+-+-+-+-+-+-+-+-+-+-+-+-+-+-+-+-+-+-+-+-+-+
      |                         Layer 4                               |
      +-+-+-+-+-+-+-+-+-+-+-+-+-+-+-+-+-+-+-+-+-+-+-+-+-+-+-+-+-+-+-+-+
      |                         Payload                               |
      +-+-+-+-+-+-+-+-+-+-+-+-+-+-+-+-+-+-+-+-+-+-+-+-+-+-+-+-+-+-+-+-+
      |                   IFA Metadata Stack                          |
      +-+-+-+-+-+-+-+-+-+-+-+-+-+-+-+-+-+-+-+-+-+-+-+-+-+-+-+-+-+-+-+-+
      |                   IFA Metadata Header                         |
      +-+-+-+-+-+-+-+-+-+-+-+-+-+-+-+-+-+-+-+-+-+-+-+-+-+-+-+-+-+-+-+-+
\end{verbatim}
\end{center}
\normalsize

A closer look at the IPv4 header demonstrates that Protocol=IFA would be
used to signal that this frame is an IFA frame:

\tiny
\begin{center}
\begin{verbatim}
       0                   1                   2                   3
       0 1 2 3 4 5 6 7 8 9 0 1 2 3 4 5 6 7 8 9 0 1 2 3 4 5 6 7 8 9 0 1
      IPv4 Header:
      +-+-+-+-+-+-+-+-+-+-+-+-+-+-+-+-+-+-+-+-+-+-+-+-+-+-+-+-+-+-+-+-+
      |Version|  IHL  |Type of Service|          Total Length         |
      +-+-+-+-+-+-+-+-+-+-+-+-+-+-+-+-+-+-+-+-+-+-+-+-+-+-+-+-+-+-+-+-+
      |         Identification        |Flags|      Fragment Offset    |
      +-+-+-+-+-+-+-+-+-+-+-+-+-+-+-+-+-+-+-+-+-+-+-+-+-+-+-+-+-+-+-+-+
      |  Time to Live | Protocol = IFA|         Header Checksum       |
      +-+-+-+-+-+-+-+-+-+-+-+-+-+-+-+-+-+-+-+-+-+-+-+-+-+-+-+-+-+-+-+-+
      |                     Source IPv4 Address                       |
      +-+-+-+-+-+-+-+-+-+-+-+-+-+-+-+-+-+-+-+-+-+-+-+-+-+-+-+-+-+-+-+-+
      |                  Destination IPv4 Address                     |
      +-+-+-+-+-+-+-+-+-+-+-+-+-+-+-+-+-+-+-+-+-+-+-+-+-+-+-+-+-+-+-+-+
\end{verbatim}
\end{center}
\normalsize

Additionally the IFA Header provides a Next Header field that would
indicate that TCP is the next protocol:

\tiny
\begin{center}
\begin{verbatim}
       0                   1                   2                   3
       0 1 2 3 4 5 6 7 8 9 0 1 2 3 4 5 6 7 8 9 0 1 2 3 4 5 6 7 8 9 0 1
      IFA Header:
      +-+-+-+-+-+-+-+-+-+-+-+-+-+-+-+-+-+-+-+-+-+-+-+-+-+-+-+-+-+-+-+-+
      |Ver=2.0|  GNS  | NextHdr = TCP |R|R|R|M|T|I|T|C|   Max Length  |
      |       |       |               | | | |F|S| |A| |               |
      +-+-+-+-+-+-+-+-+-+-+-+-+-+-+-+-+-+-+-+-+-+-+-+-+-+-+-+-+-+-+-+-+
\end{verbatim}
\end{center}
\normalsize

\section{Hardware Requirements}

Hardware requirements/implementations for handling IFA/INT today and the
challenges facing those wanting to implement them in hardware and
software.  

\section{Possible Configuration Methods}

Proposals for configuring INT/IFA in both software dataplane and
hardware dataplane environments on supported hardware.

\subsection{Host-based configuration}

\subsection{Network-based configuration}

\section{Risks and Rewards}

Assessing and minimizing risk associated with the maintenance of
deploying an early draft of a standard.

\section{Acknowledgments}

Special thanks to all those invoved in IFA and INT specs as well as
those at Broadcom who did not particiapte in the writing of the
standard, but did provide code, feedback, etc during this process.

{\normalsize \bibliographystyle{acm}
\bibliography{int}}


\theendnotes

\end{document}
